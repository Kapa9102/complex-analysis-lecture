% lec2.tex 
\lecture{2}{08:00 AM Mon, Oct 06 2025}{} 
\begin{proposition}[Ratio Test Formula]
  Let $\sum_{n=0}^{\infty} a_n (z-z_0) ^n  $ be a power series. Suppose that the limit 
  \[
  \al = \lim_{n \to \infty}
  \left| 
  \frac{a_n }{a_{n+1}}
  \right|
  \]
  exists (i.e., $\in  [0, \infty ]$). Then the radius of convergence $R $ of the power series in question is 
  $R = \al$. 
\end{proposition}
\begin{proof}
We use the d'Allembert rule for the series 
\[
\sum_{n=0}^{\infty} a_n (z-z_0) ^n \quad \quad   (z \in  \CC \backslash \left\{ z_0 \right\})  .
\]
Let $z \in \CC \backslash \left\{ z_0 \right\} $. we have: 
\begin{align*}
  \lim_{n \to \infty} \left| \frac{a_{n+1}(z-z_0) ^{n+1}}{a_n (z-z_0) ^n } \right| 
  &=
  \lim_{n \to \infty} \left| \frac{a_{n+1}}{a_n } \right| \cdot 
  \left| z-z_0 \right| \\
  &= 
  \left| z-z_0 \right|\cdot \lim_{n \to \infty} \left| \frac{a_{n+1}}{a_n } \right| \\
  &= \frac{\left| z-z_0 \right|}{\al}
\end{align*}
By the d'Allembert rule, we have: 
\begin{itemize}
  \item[\ding{49}] The series $\sum_{n=0}^{\infty} a_n (z-z_0) ^n  $ 
    converges if 
    \[
    \frac{\left| z-z_0 \right|}{\al}< 1 \quad  \text{ i.e. } 
    \quad 
    \left| z-z_0 \right|< \al.
    \]
  \item[\ding{49}]
      The series $\sum_{n=0}^{\infty} a_n (z-z_0) ^n $ 
      diverges if 
      \[
      \frac{\left| z-z_0 \right|}{\al} > 1 
      \quad 
      \text{ i.e. } \quad   \left| z-z_0 \right| > \al.
      \]
      Hence $R = \al$. 
\end{itemize}
\end{proof}
\begin{example}
Determine the radius of convergence of the power series $\sum_{n=0}^{\infty} \frac{z _n }{n!}$ where $z_0 = 0$.  \\
\textcolor{black}{
  \underline{
\uline{
\textsc{
1 $^{st} $ Method: (By Hadamard formula)
}
  }
}\\
}
We must compute $\lim_{n \to \infty} \sup_{} \left( \frac{1}{n!} \right)^{\frac{1}{n}} $. By
the stirling formula, we have that: 
\[
  n! \sim_{\text{\tiny$+\infty  $ }   } n^{n}e^{-n} \sqrt{2\pi n} .
\]
Thus we get: 
\[
(n!) ^{\frac{1}{n}} \sim_{\text{\tiny$+\infty  $ }   } n e^{-1}(2 \pi n) ^{\frac{1}{2n}}.
\]
Thus \[
\left( \frac{1}{n!} \right)^{\frac{1}{n}} 
 \sim_{\text{\tiny$+\infty  $ }   }
\frac{e}{n}(2 \pi  n) ^{-\frac{1}{2n}} \rightarrow 0 \text{ as $n \rightarrow +\infty  $ } .
\]
Thus $R = \frac{1}{0} = +\infty$. \\
This means that the power series $\sum_{n=0}^{\infty} \frac{z^{n}}{n!} $ converges
for all $z \in \CC $. \\
\newpage
  \underline{
\uline{
\textsc{
2 $^{nd} $ Method:
}
  }
}\\
We use \it Proposition $2 $ \normalfont. we have: 
\begin{align*}
\lim_{n \to \infty} \left| \frac{\frac{1}{n!}}{\frac{1}{(n+1) !}} \right|  
&= 
\lim_{n \to \infty} \frac{(n+1) !}{n!} \\
&= \lim_{n \to \infty} (n+1)  = +\infty .
\end{align*}
Thus $R = +\infty $ 
\end{example}
\section{Analytic Functions}
\begin{definition}[]
Let $\Omega  $ be a non empty open subset of $\CC  $ and let $z_0 \in   \Omega  $. \\
Let $ f : \Omega  \longrightarrow \CC  $ be a map. then: 
\begin{enumerate}
\item $f $ is said to be analytic at $z_0 $ if there exists $r > 0 $ and a complex sequence 
  $(a_n ) _{n \in   \NN_0} $ such that $D(z_0, r) \subset \Omega  $  and: 
  \[
  f(z) = 
  \sum_{n=0}^{\infty} a_n (z-z_0) ^n \quad 
  \left( 
  \forall  z \in   D(z_0, r)
  \right).
  \]
  \item $f $ is said to be analytic on $\Omega  $ if its analytic at every point of 
    $\Omega $. 
\end{enumerate}
\end{definition}
\begin{center}
\includegraphics[width=5cm]{images/analytic.png}
\end{center}
\begin{example}
  \begin{enumerate}
    \item Every complex polynomial is analytic on $\CC  $. Indeed, let $P \in  \CC [ \mathbb{Z} ]$, and 
      $z_0 \in   \CC  $. since $P(z + z_0)  \in  \CC [ \mathbb{Z}]$, we can write: 
      \[
      P(z+z_0)  = 
      \sum_{n=0}^{d} a_n z^n  \quad (d \in   \NN_0) .
      \]
      Substituting $z $ by $(z-z_0)  $, we get: 
      \[
      P(z) = 
      \sum_{n=0}^{d} a_n (z-z_0) ^n    ,
      \]
      which is a power series centered at $z_0 $ with infinite randius of convergence. 
      Thus, $P $ is analytic at $z_0 $. Since $z_0 $ was arbitrary, $P $ is analytic
      on $\CC$. 
      \divider
      \item 
        The function $z \longrightarrow  \frac{1}{z}$ is analytic on 
        $\CC ^{*} = \CC  \backslash  \left\{ 0 \right\} $. Indeed, let $z_0 \in   \CC ^{*} $ arbitrary. \\
        For $z \in  D(z_0, \left| z_0 \right|  )  $, we have: 
        \[
        \left| \frac{z-z_0}{z_0} \right|  < 1.
        \]
        We can write 
        \begin{align*}
          \frac{1}{z} &= \frac{1}{z_0 + (z-z_0) } 
          \\
          &= 
          \frac{1}{z_0} \cdot \frac{1}{1 + \frac{z-z_0}{z_0} } \\
          &= 
          \frac{1}{z_0} \cdot  
          \sum_{n=0}^{\infty} (-1) ^n  
          \left( \frac{z-z_0}{z_0} \right) ^n  \\ 
          &= 
          \sum_{n=0}^{\infty} \frac{(-1) ^n }{z_0^{n+1}} 
          (z-z_0) ^n ,
        \end{align*}
        which is a power series centered at $z_0 $, valid on $D(z_0, \left| z_0 \right|  )  $. Hence
         $z \longrightarrow \frac{1}{z} $ is analytic at $z_0 $. Since $z_0 \in   \CC ^{*} $ was
         arbitrary, then $z \longrightarrow \frac{1}{z} $ is analytic on $\CC ^{*} $.
  \end{enumerate}
\end{example}
  \subsection{Properties of Analytic Functions}
  \begin{proposition}[]
  Let $\Omega  $ be a non empty open subset of $\CC  $ and let $z_0 \in   \Omega  $. If 
  $ f,g : \Omega  \longrightarrow \CC  $ are analytic at $z_0 $, then the same is for $(f+g)  $ and 
  $(f \cdot g)  $. Moreover, if $f  $ and $g $ are represented by power series with radii of convergence $R_{f} $ and
  $R_{g} $ respectively then $(f+g)  $ and $(f \cdot  g)  $ are represented by power series 
  with radii of convergence $ \geq \min (R_{f}, R_{g})  $. 
  \end{proposition}
  \begin{proof}
  Exercise.
  \end{proof}
  \begin{corollary}[]
    Let $ \Omega  $ be a non empty open subset of $\CC  $ and let $ f,g : \Omega  \longrightarrow \CC $. If $f $ 
    and $g $ are both analytic on $\Omega  $, then the same is for $(f + g)  $ 
    and $(f \cdot g)$. 
  \end{corollary}
  \begin{proposition}[Analyticity  $ \implies $ Continuity]
    Let $\Omega  $ be a non empty open subset of $\CC  $ and let $z_0 \in  \Omega  $.
    Let also $ f : \Omega  \longrightarrow \CC  $ be a map. 
    If $f $ is analytic at $z_0 $ then $f $ is continuous at $z_0 $ 
  \end{proposition}
  \begin{proof}
  Suppose that $f $ is analytic at $z_0 $ then there exists $R > 0 $ and a complex sequence $(a_n ) _{n \in  \NN_0} $ such
  that $D(z_0, R) \subset \Omega  $  and: 
  \[
  f(z) = 
  \sum_{n=0}^{\infty} a_n (z-z_0) ^n  \quad (\forall  z \in   D(z_0, R) ) 
  \]
  In particular, $f(z_0)  = a_0 $. Thus for all $z \in   D(z_0, R)  $ we have: 
  \begin{align*}
    f(z)  - f(z_0)  &= 
    \sum_{n=1}^{\infty} a_n (z-z_0) ^n  \\
                    &= 
                    (z-z_0) \sum_{n=1}^{\infty} a_n (z-z_0) ^{n-1} \\
                    &= (z-z_0)  
                    \sum_{n=0}^{\infty} a_{n+1} (z-z_0) ^n  \quad \quad  (1)
  \end{align*}
  By the Hadamard formula, we see that the power series 
  $\sum_{n=0}^{\infty} a_{n+1}(z-z_0) ^n  $ has the same 
  radius of convergence as the original power series $\sum_{n=0}^{\infty} a_n (z-z_0) ^n  $.
  Consequently, the power series $\sum_{n=0}^{\infty} a_{n+1}(z-z_0) ^n  $  converges
  absolutely for $\left| z-z_0 \right|  < R $. Let $r \in  \RR  $ such that 
  $0 <  r <  R$. Then for all $z \in  D(z_0, r)  $, we have  from (1) the estimate: 
  \begin{align*}
    \left| f(z) - f(z_0)  \right|  &= 
    \left| z-z_0 \right|   
    \cdot 
    \left| \sum_{n=0}^{\infty} a_{n+1}(z-z_0) ^n  \right|  
                                \\ & \leq 
                                \left| z-z_0 \right|  \sum_{n=0}^{\infty} \left| a_{n+1} \right|  
                                \left| z-z_0 \right|  ^n \\
                                   & \leq 
                                   \left| z-z_0 \right|  
                                   \underbrace{
                                   \sum_{n=0}^{\infty} \left| a_{n+1} \right|  \cdot 
                                   r^{n}
                                   }_{< +\infty  \text{ since }  r <  R } .
  \end{align*}
  Taking the limit as $z \rightarrow z_0 $, we conclude that $\lim_{z \to z_0} f(z) = f(z_0)  $, so $f $ is continuous at 
  $z_0$. 
  \end{proof}
  \begin{corollary}[Immediate]
    Let $\Omega  $ be a non empty open subset of $\CC  $ and $ f : \Omega  \longrightarrow \CC  $. If 
    $f $ is analytic on $\Omega  $, then $f $ is continuous on $\Omega$. 
  \end{corollary}
  \begin{proposition}[Composition of Analytic functions]
    Let $\Omega_{1} $ and $\Omega _{2} $ be two 
    nonempty open subsets of $\CC  $ and 
    let
    $ f : \Omega _{1} \longrightarrow \Omega _{2} $ and 
    $ g : \Omega _{2} \longrightarrow \CC  $ 
    be two maps. Let also $z_0 \in  \Omega _{1} $. If $f $ is analytic
    at $z_0 $ and $g $ is analytic at $f(z_0)  $, then $(g \circ f)  $ 
    is analytic at $z_0$. 
  \end{proposition}
  \begin{proof}
  Exercise 
  \end{proof}
  \begin{corollary}[Immediate]
    Let $\Omega_{1} $ and $\Omega _{2} $ be two 
    nonempty open subsets of $\CC  $ and let $ f : \Omega _{1} \longrightarrow  \Omega _{2}$ 
    and $ g : \Omega _{2} \longrightarrow \CC  $ be two maps. If $f $ is analytic
    on $\Omega _{1} $ and $g $ is analytic on $\Omega _{2} $ then $(g \circ f)  $ is analytic
    on $\Omega _{1} $. 
  \end{corollary}
  \begin{proposition}[Quotient of Analytic Functions]
    Let $\Omega  $ be a nonempty open subsets of $\CC  $ and let $z_0 \in  \Omega  $. Let 
    also $ f,g : \Omega  \longrightarrow \CC  $ be two functions which are both analytic at $z_0 $ 
    and such that $ g(z_0) \neq 0 $. Then the function $\frac{f}{g} $ is analytic at $z_0 $.
  \end{proposition}
  \begin{proof}
  Since $g(z_0) \neq 0 $ then the function $ h : w \longrightarrow \frac{1}{w} $ is analytic at $g(z_0)  $
  (as seen in previous examples). Therefore, by \it Proposition $1.2.5$\normalfont, the function $\frac{1}{g} = h \circ g $ is 
  anayltic at $z_0 $. \\
  It then follows from \it Proposition $1.2.1 $ \normalfont that the product $f \cdot \left( \frac{1}{g} \right)  $ is analytic
  at $z_0 $.
  \end{proof}
  \begin{corollary}[Immediate]
    Let $\Omega  $ be a non empty open subset of $\CC  $ and let $ f,g : \Omega  \longrightarrow \CC  $ be
    two analytic functions on $\Omega  $ such that $g(z) \neq 0 $ for every $z \in   \Omega $. Then the
    function $\frac{f}{g} $ is analytic on $\Omega  $ .
  \end{corollary}
  \begin{example}
    Every rational function is analytic on its domain of definition. This is because a rational
    function is a quotient of two polynomials, and polynomials are analytic on $\CC  $.
  \end{example}
  \divider
  \section{Power series define Analytic functions}
  \begin{theorem}[]
  A power series with a positive radius of converges defines an analytic function on its disk of convergence.
  \end{theorem}
  \begin{proof}
  Let $\sum_{n=0}^{\infty} a_n (z-z_0) ^n  $ be a power series $(z_0 \in  \CC , (a_n ) _{n \in  \NN)} \subset \CC )  $ 
  with radius of convergence $R > 0 $. Define the function $f $ on the disk $D(z_0, R)  $ by: 
  \[
  f(z)  = \sum_{n=0}^{\infty} a_n (z-z_0) ^n .
  \]
  We must show that $f $ is analytic on $D(z_0, R)$. Let $z_1 \in D(z_0, R) $ arbitrary. We will show that $f $ 
  is analytic at $z_1 $. For $z \in   D(z_1, R - \left| z_1-z_0 \right|  )  $, we have 
  \[
  \left| z-z_0 \right|  \overset{T.I}{ \leq } 
  \underbrace{
    \left| z-z_1 \right|  
  }_
 {
   < R - \left| z_1-z_0 \right|  
 } 
 + \left| z_1-z_0 \right|  <  R
  \]
  Thus $D(z_1, R - \left| z_1-z_0 \right|  )  \subset D(z_0, R)  $, so the power series 
  $\sum_{n=0}^{\infty} a_n (z-z_0) ^n  $ converges absolutely. so: 
  \begin{align*}
    f(z) &= \sum_{n=0}^{\infty} a_n (z-z_0) ^n \\
         &= 
         \sum_{n=0}^{\infty} a_n ((z-z_1)  + (z_1 - z_0) ) ^n 
         \\
         &= 
         \sum_{n=0}^{\infty} a_n \sum_{k=0}^{n}  
         \binom{n}{k} 
         (z-z_1) ^{k} (z_1-z_0) ^{n-k}
         \\
         &= 
         \sum_{k=0}^{\infty} \left( 
           \sum_{n=k}^{\infty } a_{k}
           \binom{ n}{k} 
           (z_1 - z_0) ^{n-k}
         \right) 
         (z-z_1) ^{k}
  \end{align*}
  The interchange of summation is justified by the absolute convergence of the double
  series for $z \in   D(z_1, R - \left| z_1- z_0 \right|  )  $. 
  This express $f(z)  $ as a power series in $(z-z_1)  $ in the disk $D(z_1, R - \left| z_1-z_0 \right|  )  $, proving
  that $f $ is analytic at $z_1 $. Since $z_1 $ was arbitrary in $D(z_0, R)  $, then $f $ is analytic on 
  $D(z_0, R)$. 
  \end{proof}
%  end of file
