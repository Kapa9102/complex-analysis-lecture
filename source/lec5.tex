% lec5.tex 
\lecture{5}{14:41 PM Tue, Oct 21 2025}{} 
\begin{corollary}[Cauchy-Riemann equations on an open set]
  Let $\Omega$ be an open subset of $\CC $ and $ f : \Omega \longrightarrow \CC  $ be a map. Let 
  $P := \text{Re}  f :  \Omega\longrightarrow  \RR $ and $Q := \text{Im}  f :  \Omega\longrightarrow  \RR $, so that
  \[
  f(z)  = P(x, y)  + i Q(x, y)
  \]
  for all $z = x + iy \in  \Omega$, with $x,y \in   \RR $. Then $f$ is holomorphic on $\Omega$ if and only if $P$ 
  and $Q$ are differentiable on $\Omega$ and satisfy the following Cauchy-Riemann equations:
  \begin{align*}
    \frac{\partial P}{\partial x}  &= \frac{\partial Q}{\partial y}, \\
    \frac{\partial P}{\partial y}  &= - \frac{\partial Q}{\partial x},
  \end{align*}
  on $\Omega$.
\end{corollary}
\subsection{The isolated zeros thereom}
\noindent \textcolor{larratBicep!10!brown}{
  \uline{ 
    \large \emph{ \ding{43} Some toplogical remainders:}
  }
}\\
\begin{definition}[Limit points]
  Let $E$ be a toplogical space,
  $A \subset E$, $x \in   E$. we say that $x$ is
  a \underline{limit point} of $A$ if every
  neighborhood of $x$ intersect $A$ in a point 
  different from $x$; That is, 
  \[
  \forall V \in   \mathcal{V} (x) : 
  \quad V \cap \left( A \backslash  \left\{ x \right\} \right) 
  \neq  \emptyset .
  \]
  Note that this is equivallent to $x \in  \overline{A \backslash  \left\{ x \right\}}$. The set
  of all limit points of $A$ is denoted by $A'$ and is called the derived set of $A$.
\end{definition}
\ding{49} In metric spaces, we have the following equivallent definition:
\begin{definition}[]
  Let $E$ be a metric space, $A \subset E$, and $x \in   E$. We say that $x$ 
  is a \underline{limit point} of $A$ if there exists a sequence $(x_n ) _{n \in  \NN}$ 
  in $A \backslash  \left\{ x \right\}$ that converges to $x$ .
\end{definition}
\divider
\begin{definition}[]
Let $E$ be a toplogical space and $A \subset E$. 
\begin{itemize}
  \item[\ding{172} ] We say that an element $a \in   A$ is \underline{isolated} if its not a limit point of 
    $A$.
  \item [\ding{173} ] We say that $A$ is a discrete set if all its points are isolated.
\end{itemize}
\end{definition}
\begin{example}
The set $\NN$ and $\mathbb{Z}$ are discete in $\RR $, whereas the set $\mathbb{Q}$ is not (even though its countable).
\end{example}
\begin{proposition}[]
In $\RR ^n (n \in  \NN)  $, every discrete set is at most countable.
\end{proposition}
\begin{proof}
  \ding{49} Exercise !
\end{proof}
\begin{theorem}[The isolated zeros theorem]
  Let $\Omega$ be a nonempty connected open set in $\CC $ and let $f$ be a an analytic function on $\Omega$ that is 
  not identically zero. Then the zero of $f$ in $\Omega$ are all isolated; In other words, the set of zeros of $f$ in
  $\Omega$ is discrete.
\end{theorem}
\begin{proof}
We proceed by contradiction. Suppose that there exists a zero $z_0 \in   \Omega$ of $f$ that is not isolated; 
i.e., $z_0$ is a limit point of the set of all zeros of $f$ in $\Omega$. Therefore, there exists a sequence 
$(z_k ) _{k \geq 1}$ of zeros of $f$ in $\Omega$, with all terms distinct from $z_0$, that converges to $z_0$. Since
$f$ is analytic at $z_0$, $\exists  r > 0$ and a power series representation:
\[
f(z) = \sum_{n=0}^{+\infty} a_n (z-z_0) ^n \quad \quad (\forall  z \in   D(z_0, r) ) 
\]
with $a_n  \in   \CC $ for all $n \in  \NN_0$.
\begin{itemize}
  \item [\ding{50}\ding{172} ] Let us first show that the coefficients $ a _n (n \in  \NN) $ must necessarily
    all be zero. We proceed by contradictiom, assuming the contrary, and consider
    \[
    P := \min \left\{ n \in  \NN_0: a_n \neq 0 \right\}.
    \]
    We then have for all $z \in   D(z_0, r):$ 
    \begin{align*}
      f(z)  &= 
      \sum_{n=p}^{+\infty } 
      a_n (z-z_0) ^n  \\
            &= (z-z_0) ^{p} \left[ a_{p} + a_{p+1}(z-z_0)  + a_{p+2}(z-z_0) ^2  \right] \\
            &= (z-z_0) ^{p} \sum_{n=0}^{+\infty} a_{n+p} (z-z_0) ^n.
    \end{align*}
    By specializing to $z = z_k $ (for $k$ sufficiently large so that $z_k \in  D(z_0, r)$) we find that 
    (for $k \geq 1$ sufficiently large): 
    \[
    \underbrace{f(z_k )
    }_{=0}  = \underbrace{(z_k  - z_0)^{p}
    }_{\neq 0}   \sum_{n=0}^{+\infty} a_{n+p}(z_k -z_0) ^n.
    \]
    Hence
    \[
    \sum_{n=0}^{+\infty} a_{n+p} (z_k -z_0) ^n = 0 ,
    \]
    taking the limit as $k \rightarrow +\infty$ and noting the normal convergence 
    of the series on the left, we obtain: $a_{p} = 0.$  This contradicts the definition of $p$ and shows that 
    $a_n = 0$ for all $n \in  \NN_0$. In follows from this that: 
    \[
    f(z)  = 0 \quad \quad (\forall  z \in   D(z_0, r) ).
    \]
  \item [ \ding{50} \ding{173} ] Now, we will show that $f$ is identically zero on all $\Omega$, which 
    will yield the desired contradiction. Let $\omega \in  \Omega$ be arbitrary and show that $f(\omega) =  0.$ 
    Since $\Omega$ is connected (no path-connected, as its an open subset of $\CC $), there exists a continuous 
    path $ \gamma  : [0,1] \longrightarrow \Omega $ such that $\gamma (0) = z_0$ and $\gamma (1)  = \omega$. Consider
    \[
    t_0 := \sup \left\{ t \in   \left[ 0, 1 \right]: (f \circ  \gamma ) (t) = 0 \right\} 
    \]
    The supremum exists because the set is non empty, as it contains $0$, and it is bounded from above by $1$. Since $f$ 
    and $\gamma $ are continuous, the function $f \circ \gamma $  are continuous on $\left[ 0, 1 \right]$. Consequently, 
    the set 
    \[
      \left\{ t \in  [0, 1]: (f \circ \gamma ) (t) =  0 \right\} = 
      \left( f \circ  \gamma  \right) ^{-1} \left( \left\{ 0 \right\} \right) 
    \]
    which is closed in $[0, 1]$. Thereforem $t_0$ belongs to this set; in other words, we have
    \[
    \left( f \circ \gamma  \right) (t_0) = 0 \quad \quad (1) 
    \]
    Let us show that $t_0  = 1$. Suppose for contradiction, that $t_0 <  1$. By the reasoning from the previous 
    part of this proof (replacing $z_0 $ by $\gamma (t_0)$, which is a limit point of the zero of $f$), 
    there exists $r' > 0$ such that 
    \[
    f(z)  = 0 \quad \quad (\forall  z \in   D(\gamma (t_0), r') ) .
    \]
    FOr $\veps  > 0$, sufficiently small, we have: 
    \[
    \gamma (t_0 + \veps ) \in  D(\gamma (t_0), r'),
    \]
    (by the continuity of $\gamma $). Therefore: 
    \[
    g(\gamma (t_0 + \veps ) ) = 0,
    \]
    i.e. $(f \circ  \gamma ) ( t_0  + \veps )  = 0.$
    This contradicts the very definition of $t_0$ as the supremum. Hence, necessarily $t_0 = 1$. This gives, from 
    $(1) $, $f (\omega) = 0$. Since $\omega$ was arbitrary in $\Omega$, we have $f \equiv  0$ on $\Omega$. Contradiction. This 
    final contradiction ensures that the zeros of $f$ in $\Omega$ are all isolated. The theorem is proved. 
    \texttt{there is a tiny error in this proof, will be fixed next time.}
\end{itemize}
\end{proof}
\begin{corollary}[Principle of analytic continuation]
  Let $f$ and $g$ be two analytic functions on a nonempty \underline{connected} open subset 
  $\Omega$ of $\CC $ that coincide on a subset $A \subset \Omega$ pocessing a limit point in $\Omega$. Then 
  $f$ and $g$ are identical on $\Omega$.
\end{corollary}
\begin{proof}
Let $\FF := f - g$. Then $\FF$ is analytic on $\Omega$ and vanishes on the set $A \subset \Omega$, which has a limit
point $a \in  \Omega$. So $a \in  \overline{A \backslash  \left\{ a \right\}} \subset \overline{A}$. Since 
$\FF$ vanishes on $A$ and is continuous on $\Omega$, then it vanishes on $\overline{A} \cap \Omega$. In particular, 
$\FF$ vanishes at $a.$ Therefore, $a$ is a non-isolated
zero of $\FF$. By the isolated zero theorem, this implies
that $\FF \equiv 0$ on $\Omega$; That is, 
$f \equiv g$ on $\Omega$.
\end{proof}
\begin{example}
Let us show that (without using the extended Euler
formulas) that for all $z \in   \CC $, we have: 
\[
\cos ^2  z + \sin  ^2  z = 1.    
\]
consider $f(z)  := \cos ^2  z + \sin ^2  z    $ and $g(z) := 1$. $f$ and $g$ are analytic 
on $\CC $ (which is an connected open subset of $\CC $) and coincide on $\RR $, which pocesses a limit point 
in $\CC $. Thus, by the principle of analytic continuation $f \equiv g$ on $\CC $; i.e. 
\[
\cos ^2 z + \sin ^2  z = 1 \quad \quad (\forall  z \in   \CC ) .
\]
\end{example}
\subsection{Multiplicity of a zero of an analytic function}
\begin{theorem}[]
  Let $\Omega$ be a nonempty connected open subset of $\CC $,
  and let $f$ be an analytic function on $\Omega$, not identically zero. Let $z_0 \in   \Omega$ be a zero of $f$. Then there exists a unique
  positive integer $p$ and a unique analytic function 
  $g$ on $\Omega$ does not vanish at $z_0$, such that
  \[
  f(z)  = (z-z_0) ^{p}g(z)  \quad \quad 
  (\forall  z \in   \Omega) 
  \]
\end{theorem}
\begin{proof}
\underline{\ding{50} \textbf{Existence of $p$ and $g$ :}} \\
Since $f$ is analytic at $z_0$, $\exists  r > 0$ and a complex sequence $(a_n ) _{n \in  \NN_0}$ such that 
$D(z_0, r ) \subset \Omega$  and 
\[
f(z)  = \sum_{n=0}^{+\infty} a_n (z-z_0) ^n \quad \quad (\forall  z \in   D(z_0, r) ).
\]
Since $f$ is not identically zero on $\Omega$, it is certainly not identically zero on $D(z_0, r)$ (By the isolated zeros 
theorem). Thus the coefficients $a_n $ are not all zero. We can therefore define 
\[
p := \min \left\{ n \in  \NN_0: a_n  \neq 0 \right\}.
\]
since $a_0 = f(z_0) $, we have $p \geq 1$. Then, for all $z \in D(z_0, r)$: 
\begin{align*}
  f(z)  &= \sum_{n=p}^{+\infty} a_n (z-z_0) ^n   \\
        &= (z-z_0) ^{p} \sum_{n=0}^{+\infty} a_{n+p} (z-z_0) ^n 
\end{align*}
Now, define $ g : \Omega \longrightarrow \CC  $ by: 
\[
g(z)  := 
\begin{cases}
  \frac{f(z) }{(z-z_0) ^{p}} & \text{ if }  z \neq  z_0,\\
  a_{p} & \text{ if }  z = z_0.
\end{cases}
\]
We observe that:
\begin{itemize}
  \item $g$ is analytic on $\Omega \backslash  \left\{ z_0 \right\}$ (as a quotitent of two 
    analytic functions on $\Omega \backslash  \left\{ z_0 \right\})$.  
    \item For $z \in   D(z_0, r) $, $g(z) = \sum_{n=0}^{+\infty} a_{n+p}(z-z_0) ^n $ which shows that $g$ 
      is analytic at $z_0$. Hence, $g$ is analytic on $\Omega$. Moreover, we have 
      \[
      f(z)  = (z-z_0) ^{p} g(z)  \quad \quad (\forall  z \in   \Omega) 
      \]
      and $g(z_0)  = a_{p} \neq 0 $ .
\end{itemize}
\underline{\ding{50} \textbf{Uniqueness of $p$ and $g$ :}} \\
Suppose there exists $p_1, p_2 \in  \NN$ and analytic functions $g_1, g_2 $ on $\Omega$ that do not vanish 
at $z_0$, such that 
\[
f(z)  = (z-z_0) ^{p_1} g_1(z) = (z-z_0) ^{p_2}  g_2(z)   \quad \quad 
(\forall  z \in   \Omega) .
\]
Then, for all $z \in   \Omega \backslash  \left\{ z_0 \right\}$, we have: 
\[
g_1(z)  = 
(z-z_0) ^{p_2 - p_1} g_2(z) .
\]
If $p_1 <  p_2$ then taking the limit as $z \rightarrow z_0$ and using the continuity of $g_1$ and $g_2$ at $z_0$, 
we obtain $g_1(z_0)  = 0$, which contradicts the hypothesis $g_1(z_0)  \neq 0$. Therefore, we must have 
$p_1 \geq p_2$. By symmetry, we also have $p_2 \geq p_1$, so $p_1 = p_2$. Then, from above, we get 
\[
  \forall z \in   \Omega \backslash  \left\{ z_0 \right\}: 
  g_1(z)  = (z-z_0) ^{p_2-p_1} g_2(z),
\]
since $p_2 - p_1 = 0$, we get
\[
g_1(z) = g_2(z).
\]
Since $g_1$ and $g_2$ are continuous at $z_0$, taking the limit as $z \rightarrow z_0$ gives 
$g_1(z_0)  = g_2(z_0) $. Hence $g_1 \equiv g_2$  on $\Omega$, which completes the proof of uniqueness.
\end{proof}
\begin{definition}[]
In the context of the above theorem, the positive integer $p$ is called the 
\underline{multiplicity} of the zero $z_0$ of $f$. if $z_1$ is not a zero of $f$, its multiplicity 
is conventionally taken to be $0$.
\end{definition}
% end of file
