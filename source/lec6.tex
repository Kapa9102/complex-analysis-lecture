% lec6.tex 
\chapter{The Cauchy integral formula on a circle and applications}
\lecture{6}{08:09 AM Mon, Oct 27 2025}{} 
\begin{theorem}[]
Let $z_0 \in   \CC $ and $ f : \CC  \longrightarrow \CC  $ be
a function, analytic at $z_0$. Let also $R$ be the radius
of convergence of the power series representing $f$ 
in a neighborhood of $z_0$. Then for every 
$n \in  \NN_0$ and every $ 0 <  r <  R$, we have:
\begin{center}
\begin{tcolorbox}[boxrule=1pt, colback=larratBicep!60, sharp corners, width=7cm, height=2cm, top=-0.1cm]
\[
f^{(n) }(z_0)  = 
\frac{n!}{2 \pi  r^n }
\int_{0}^{2\pi } 
f(z_0 + r e^{i \theta}) 
e^{-n i \theta} d\theta
\]
\end{tcolorbox}
\end{center}
In particular, for $n = 0$ and $0 <  r <  R$.
\begin{center}
\begin{tcolorbox}[boxrule=1pt, colback=larratBicep!60, sharp corners, width=7cm, height=2cm, top=-0.1cm]
\[
  f(z_0)  = 
  \frac{1}{2\pi }
  \int_{0}^{2\pi } 
  f(z_0 + r e^{i \theta}) d \theta
\]
\end{tcolorbox}
\end{center}
\end{theorem}
\begin{proof}
By hypothesis, $\exists !$ complex sequence $(a_n ) _{n \in  \NN_0}$ such that for all $z \in  D(z_0, R) $ 
\[
f(z)  = 
\sum_{n=0}^{+\infty} a_n (z-z_0) ^n \quad \quad (\forall  z \in   D(z_0, R)) 
\]
Fix $r \in  (0, R) $. For every $\theta \in   \left[ 0, 2\pi  \right]$, taking $z = z_0 + r e^{i \theta} \in  
C(z_0, r)  \subset D(z_0, R) $ in the formula above, we obtain
\[
f(z_0 + r e^{i \theta}) = 
\sum_{n=0}^{+\infty} a_n r^n e^{i n \theta}.
\]
Since the trigonometric series on the right converges normally (so uniformally) with respect
to $\theta \in   \left[ 0, 2\pi  \right]$ (Because for all $n \in  \NN_0$, we have
  $
\left| a_n  r^n  e^{n i \theta} \right|  =
\left| a_n r^n  \right|  
  $, and the series $\sum_{n=0}^{+\infty} \left| a_n  \right|  r^n $ converges by properties of power series) then that is
  the Fourier series of the function $\theta \mapsto f(z_0 + r e^{i \theta}) $. Consequently, for 
  every $n \in  \NN_0$, the Fourier coefficients are given by:
  \[
  a_n r^n  = \frac{1}{2\pi }
  \int_{0}^{2\pi } f(z_0 + r e^{i \theta}) e^{-n i \theta} d \theta .
  \]
  Thus 
  \[
  a_n = \frac{1}{2\pi r^n }
  \int_{0}^{2\pi } 
  f(z_0 + r e^{i\theta})  e^{-ni\theta}d\theta.
  \]
  Comparing this with Taylor's formula 
  \[
    a_n  = \frac{f^{(n) } (z_0) }{n!}
  \]
  , we obtain for every $n \in  \NN_0$:
  \[
  f^{(n) }(z_0)  = 
  \frac{n!}{2 \pi  r^n }\int_{0}^{2\pi } 
  f(z_0 + r e^{i \theta}) e^{-in \theta}d \theta,
  \]
  as required.
\end{proof}
\begin{remark}
  In the context of \it Theorem $1$\normalfont, the right-hand side of the second formula (corresponding 
  to $n = 0$):
  \[
  f(z_0)  = \frac{1}{2 \pi }
  \int_{0}^{2\pi } 
  f(z_0 + r e^{i \theta}) d \theta.
  \]
  is precisely the average value of $f$ on the circle $C(z_0, r) $. Therefore, according to 
  \it Theorem 1\normalfont, the average value of $f$ on any circle centered at $z_0$ and
  contained in $D(z_0, R) $ is equal to the value of $f$ at $z_0.$ 
\end{remark}
\subsection{Analytic continuation of power series}
\newpage
\begin{theorem}[]
Let $\Omega$ be a nonempty open subset of $\CC $, and let $ f : \Omega  \longrightarrow \CC  $ 
be an analytic function on $\Omega$. Let $z_0 \in  \Omega$, and let 
$\sum_{n=0}^{+\infty} a_n (z-z_0) ^n $ (where $a_n \in  \CC $, for all $n \in  \NN_0$) be the 
unique power series representing $f$ in a small neighborhood 
$D(z_0, r) \subset  \Omega$  of $z_0$ with (with $r > 0$). Then, the power series
$\sum_{n=0}^{+\infty} a_n (z-z_0) ^n $ converges on every disk centered at $z_0$ and contained in $\Omega$, 
and it sum remains $f(z).$ 
\end{theorem}
\begin{proof}
We have
\[
f(z)  = 
\sum_{n=0}^{+\infty} a_n (z-z_0) ^n  \quad \quad (\forall  z \in   D(z_0, r) ) .
\]
Let $R > r$ such that $D(z_0, R) \subset \Omega$ and show that the above formula remains valid
on $D(z_0, R).$ Given $n \in  \NN_0$, by Cauchy's integral formula on a circle, we have for all 
$0 <  \sigma  <  r:$ 
\[
a_n = \frac{1}{2\pi  \sigma ^n   }
\int_{0}^{2\pi } f(z_0 + \sigma  e^{i \theta}  ) e^{-n i \theta}d \theta.
\]
hence 
$\sigma \mapsto  \frac{1}{2\pi  \sigma ^n   } \int_{0}^{2\pi } f(z_0 + \sigma  e^{i \theta}  ) e^{-n i \theta}d \theta$ is
constant (with respect to $\sigma   $) in $(0, r)$. This suggests us to define 
\[
\begin{array}{cccl}
      \FF : &  (0, R)  & \longrightarrow & \CC  \\

           &  \sigma     & \longmapsto     & \FF(\sigma   ) = 
           \frac{1}{2\pi  \sigma ^n   } \int_{0}^{2\pi } f(z_0 + \sigma e^{i \theta}  ) e^{-ni\theta} d\theta.\\ 
\end{array}
\]
$(\FF \text{ is constant in $(0, r) $. with value $a_n$ } ). $  Since $f$ is analytic (so continuous) on 
$D(z_0, R) \subset \Omega$  then $\FF$ is well-defined and differentiable on $(0, R) $. We now show that 
$\FF$ remains constant on $(0, R)$ on $(0, R).$ By applying results from measure theory 
and integration (which allow differentiation under the integral sign), we have for all $\sigma  \in  (0, R):$ 
\begin{align*}
  \FF'(\sigma   ) 
  &= - \frac{n \sigma ^{-n-1}  }{2\pi } \int_{0}^{2\pi } f(z_0 + \sigma e^{i \theta}  ) e^{-n i \theta}d \theta + 
  \frac{\sigma ^{-n}  }{2\pi }
  \int_{0 }^{2\pi } f'(z_0 + \sigma  e^{i \theta}  ) e^{(-n + 1) i \theta} d \theta. \\
  &= \frac{\sigma ^{-n-1}  }{2 \pi }
  \int_{0}^{2\pi } 
  \underbrace{
  \left[ -n f(z_0 + \sigma  e^{i \theta}  ) e^{-ni\theta} + 
  \sigma  f'(z_0 + \sigma  e^{i\theta}  ) e^{(-n + 1) i \theta}  \right]
  }_{\frac{d }{d \theta} ( -i f (z_0 + \sigma  e^{i \theta}  ) e^{-n i \theta}) } 
  d 
  \theta \\
  &= 
  \frac{\sigma ^{-n-1}  }{2 \pi  }
    \underbrace{
  \left[ 
  -if(z_0 + \sigma e^{i \theta}  ) e^{-n i \theta} 
  \right]_{0}^{2\pi } 
    }_{ = 0} 
  = 0.
\end{align*}
This shows that $\FF$ is constant on $(0, R)$, and its value is $a_n $ (since $\FF \equiv a_n $ on $(0, r)$). Thus,
we have shown that:
\[
a_n  = \frac{1}{2\pi \sigma ^n   } \int_{
0}^{2\pi } f(z_0 + \sigma  e^{i \theta}  ) e^{-n i \theta} d \theta
\quad \quad 
( \forall  n \in  \NN_0, \forall \sigma  \in  (0, R)  ) 
\]
This formula allows us to estimate $\left| a_n  \right|  $ in terms of $n$. Giving $n \in  \NN_0, \sigma  \in  (0, R)$, and 
setting $M(f)  := \sup_{z \in   C(z_0, \sigma   ) } \left| f(z)  \right|$, we have:
\begin{align*}
  a_n  &= 
  \left| \frac{1}{2\pi \sigma ^n   }\int_{0}^{2\pi } 
  f(z_0 + \sigma  e^{i\theta}  ) e^{-n i \theta}d \theta\right|   \\
       & \leq 
       \frac{1}{2\pi  \sigma ^n   } 
       \int_{0}^{2\pi } 
       \underbrace{
         \left| f(z_0 + \sigma e^{i \theta}  )  \right|  
       }_{ \leq  M(f) } d \theta\\
        \left| a_n  \right|  & \leq \frac{M(f) }{\sigma ^n   } \quad 
        \forall n \in  \NN_0, \forall \sigma  \in  (0, R)   
\end{align*}
So $\forall  z \in  D(z_0, \sigma  ): $ 
\[
\left| a_n (z-z_0) ^n  \right|  \leq 
M(f) 
\left| \frac{z-z_0}{\sigma   } \right|  ^n 
\]
implying that the power series $\sum_{n=0}^{+\infty} a_n (z-z_0) ^n $ 
converges absolutely on $D(z_0, \sigma   ) , \forall \sigma  \in   (0, R)$. Consequently, 
$\sum_{n=0}^{+\infty} a_n (z-z_0) ^n $ converges (absolutely) on $D(z_0, R)$.  
Further, define
\[
g(z) = 
\sum_{n=0}^{+\infty} a_n (z-z_0) ^n  \quad 
\quad (\forall  z \in   D(z_0, R) ) .
\]
Since $g$ is analytic on $D(z_0, R) $ and coincides with $f$ on $D(z_0, r) $, by the principle
of analytic continuation, $g$ coincides with $f$ on $D(z_0, R) $. Hence
\[
f(z)  = 
\sum_{n=0}^{+\infty} a_n (z-z_0) ^n  \quad \quad 
(\forall  z \in  D(z_0, R) ),
\]
completing the proof.
\end{proof}
\begin{corollary}[]
Let $\Omega$ be a nonempty open subset of $\CC $ and $z_0$ be a point in $\Omega$. Let also
$ f : \Omega \longrightarrow \CC  $ be an analytic function on $\Omega$ and 
$\sum_{n=0}^{+\infty} a_n (z-z_0) ^n $ (where $(a_n ) _{n \in  \NN_0} \subset \CC $) be a power series
representing $f$ in a neighborhood of $z_0$. Then the radius of convergence $R$ of that power series satisfies;
\[
R \geq  \text{dist}  (z_0, \CC \backslash \Omega).
\]
\end{corollary}
\begin{proof}
Set $\sigma := \text{dist}  (z_0, \CC  \backslash  \Omega) $. By definition of $S$, we have $D(z_0, \sigma   ) \subset \Omega$.
(Indeed, if $z \in  \CC \backslash \Omega$ then $d(z, z_0) \geq \inf_{w \in  \CC \backslash \Omega} d(w, z_0) = d
(z_0, \CC \backslash \Omega) = \sigma    $, thus $z \in  \CC \backslash D(z_0, \sigma  ) $ ). Thus, by \it Theorem 2\normalfont, 
the power series $\sum_{n=0}^{+\infty} a_n (z-z_0) ^n $ converges on $D(z_0, \sigma   ) $. Hence $R \geq \sigma $, as required.
\end{proof}
\begin{example}
  \ding{43} \textcolor{blue}{\textsc{Very Important}}.\\
  Consider the function  $f(z) = \frac{z}{e^{z} - 1}$ with the convention $f(0)  = 1$. For all $z$ in a small
  neighborhood of $\theta$, with $z \neq 0$, we have:
  \begin{align*}
    f(z) &= \frac{z}{z + \frac{z^2 }{2!} + \frac{z^3}{3!} + \hdots } \\
         &= 
         \frac{1}{1 + \frac{z}{2!} + \frac{z^2 }{3!} + \hdots },
  \end{align*}
  Which also holds for $z = 0$, showing that $f$ is analytic at $0$ (as a quotient of two analytic functions
  at $0$, with the denominator nonzero at $0$). Moreover, the zero of $z \mapsto  e^{z} - 1$ are the complex numbers 
  $2 k \pi i$ with $k \in  \mathbb{Z}$, implying that $f$ is analytic on $\CC  \backslash  2 \pi  i \mathbb{Z}$ (as
  a quotient of two analytic functions on this domain, with the denominator nonzero at any point 
  of the domain). In conclusion, $f$ is analytic on the domain 
  \begin{align*}
    \Omega &= 
    \left( \CC  \backslash  2\pi \mathbb{Z} \right) \cup \left\{ 0 \right\} \\
           &= \CC  \backslash  2 \pi i \mathbb{Z}^{*}.
  \end{align*}
  Consider the power series expansion of $f$ at $0:$ 
  \[
  \frac{z}{e^{z} - 1} = 
  \sum_{n=0}^{+\infty} B_n \frac{z^n }{n!}, \quad \quad (*) 
  \]
  where $B_n (n \in  \NN_0) $ are the famous Bernoulli numbers, which appear in nearly all branches of mathematics
  (number theory, analysis, combinatorics, algebraic geometry, etc...).:w
  \[
  B_0 = 1, B_1 = -\frac{1}{2}, B_2 = \frac{1}{6}, B_3 = 0, 
  B_4= \frac{1}{42}, \hdots 
  \]
  \[
    (B + 1) ^n  = B^n  \quad \quad (\forall  n \geq \NN_2).
  \]
  for $n = 2:$ 
  \[
  B^2  + 2B + 1 = B_2 \implies B_1 = -\frac{1}{2}
  \]
  for $n = 3:$ 
  \[
    (B + 1) ^3  = B^3  \implies 
    3 B_2 = \frac{1}{2} \implies B_2 = \frac{1}{6}.
  \]
  \it Theorem $2$ \normalfont shows that fomula $(*) $ is valid on the largest disk centered
  at $0$ and contained in $\Omega$, which is $D(0, 2\pi). $ Hence
  \[
    \frac{z}{e^{z} - 1} = 
    \sum_{n=0}^{+\infty} B_n \frac{z^n }{n!} \quad \quad (\forall z \in  \CC, \left| z \right|  < 2\pi   ).
  \]
\end{example}
\begin{remark}
  It can be shown that 
  \[
  B_{2n+1} = 0 \quad \forall n \geq 1,
  \]
  and 
  \[
  B_{2n} \sim_{+\infty } (-1) ^{n-1} \cdot \frac{2 (2n) !}{(2\pi ) ^{2n}}.
  \]
  This allows us to directly verify (using Hadamard formula) that the radius of convergence of the power series $\sum_{n=0}^{+\infty} B _n \frac{z^n }{n!}$ is exactly $2\pi .$ 
\end{remark}
% end of file
