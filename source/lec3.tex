% lec4.tex 
\lecture{3}{08:14 AM Mon, Oct 13 2025}{} 
\begin{example}
The power series $\sum_{n=0}^{\infty} \frac{z^n }{n!}$ has 
radius of convergence $R = +\infty  $. Therefore (by the previous Theorem), it defines an analytic function on the whole complex plane $\CC  $. 
\end{example}
\begin{definition}[]
The analytic function on $\CC  $ defined by: 
\[
\exp{(z) }  = e^{z} := \sum_{n=0}^{\infty} \frac{z^n }{n!}
\]
is called \underline{the exponential function.}
\end{definition}
\begin{definition}[Entire function]
  A complex function $ f : \CC  \longrightarrow \CC  $ which
  is analytic on \underline{the whole complex plane $\CC  $} is called 
  an \underline{\textcolor{purple}{entire function.}}
\end{definition}
\begin{example}
  \begin{itemize}
    \item[\ding{172}] Every complex 
      polynomial is an \underline{entire function}.
    \item[\ding{173}] The exponential 
      function $\exp (z)    $ is an 
      \underline{entire function}.
  \end{itemize}
\end{example}
\subsection{Properties of the exponential function}
\begin{proposition}[]
The exponential function defines the following properties:
\begin{enumerate}
  \item[\ding{172}] $\forall z_1, z_2 \in   \CC$, we have:
    \[
    e^{z_1 + z_2} = e^{z_1} \cdot  e^{z_2} 
    \text{  and  }  
    e^{z_1 - z_2} = \frac{e^{z_1}}{e^{z_2}}.
    \]
  \item[\ding{173}] for all $z \in   \CC  $, we have $e^{z} \neq  0$. 
  \item[\ding{174}] \textcolor{purple}{\textsc{(Euler's formula):} } $\forall  \theta \in   \RR$, we have:
     \[
     e^{i\theta} = \cos \theta + i \sin \theta    .
     \]
   \item[\ding{175}] $\forall  z \in  \CC  $, we have: 
     \[
     e^{z} = 1 \iff z \in  2 \pi i \mathbb{Z}.
     \]
     More generally, for all $z, z' \in  \CC$, we have: 
     \[
     e^{z} = e^{z'} \iff z - z' \in  2 \pi  \mathbb{Z}.
     \]
     So, the exponential function is periodic with period $2 \pi  i $.
\end{enumerate}
\end{proposition}
\begin{proof}
\item[\ding{50} \ding{172}] $\forall z_1, z_2 \in   \CC  $, we have
  \begin{align*}
    e^{z_1} \cdot  e^{z_2} &= \sum_{k=0}^{+\infty } \frac{z_1 ^{k}}{k!} 
  \cdot \sum_{\ell = 0}^{ +\infty }  
  \frac{z_2^{\ell }}{\ell !} \\ 
                           &=
                           \sum_{k, \ell \in   \NN_0}^{} 
                           \frac{z_1^{k}z_2^{\ell }}{k! \ell !} \\
                           &=
                           \sum_{n=0}^{+\infty} 
                           \left( 
                             \sum_{k, \ell \in  \NN_0, k + \ell  = n}^{} 
                             \frac{z_1^{k} z_2^{\ell }}{k! \ell !}
                           \right)  \\
                           &= 
                           \sum_{n=0}^{+\infty} \left( 
                             \sum_{k=0}^{n} 
                             \frac{z_1^{k}z_2^{n-k}}{k!(n-k)!} 
                           \right) \\
                           &= 
                           \sum_{n=0}^{+\infty} 
                           \frac{1}{n!}
                           \underbrace{
                           \left( 
                             \sum_{k=0}^{n} 
                             \frac{n!}{k!(n-k) !} 
                             z_1^{k} z_2^{n-k}
                           \right) 
                           }_{ = \sum_{k=0}^{n} \binom{n}{k} z_1^{k} z_2^{n-k} =(z_1 + z_2)^{n} }  
                           \\
                           &= 
                           \sum_{n=0}^{+\infty} \frac{1}{n!}(z_1 + z_2) ^n = e^{z_1 + z_2},
  \end{align*}
next, we have: 
\[
  e^{z_1 - z_2} \cdot e^{z_2} \overset{\text{\tiny by the first formula}  }{ = } e^{z_1 - z_2 + z_2} = e^{z_1}.
\]
Hence $e^{z_1 - z_2} = \frac{e^{z_1}}{e^{z_2}},$ as required. 
 \item[\ding{50} \ding{173}] 
   For all $z \in   \CC  $, we have:
   \[
   e^{z} \cdot e^{-z} \overset{(1) }{=}  e^{z - z} = e^{0} = 1.
   \]
   Thus $e^{z} \neq 0 $.
   \item[\ding{50} \ding{174}] \textcolor{purple}{\textsc{(Euler's Formula).}}  \\
     For all $\theta \in  \RR  $, we have:
     \begin{align*}
       e^{i \theta} &= \sum_{n=0}^{+\infty} \frac{(i \theta) ^n }{n!} \\
                    &= \sum_{n=0}^{+\infty} i^n \frac{\theta^n }{n!}
     \\
     &= 
     \sum_{n \in  \NN_0, \text{ $n $ \tiny is even}  }^{} 
     i^n \frac{\theta^n }{n!} + 
     \sum_{n \in  \NN_0, \text{ $n $ \tiny is odd}  }^{} 
     i^n \frac{\theta ^n }{n!} \\
     &= \sum_{k=0}^{ +\infty } i^{2k} \frac{\theta^{2k}}{(2k) !} + 
     \sum_{k=0}^{+\infty } i^{2k+1} \frac{\theta^{2k+1}}{(2k+1) !}
     \\
     &=
     \underbrace{
     \sum_{k=0}^{+\infty } 
     (-1) ^{k} \frac{\theta^{2k}}{(2k)! }
     }_{\cos \theta  }  + i 
     \underbrace{
     \sum_{k=0}^{+\infty } (-1) ^{k} \frac{\theta^{2k+1}}{(2k+1)! }}_{\sin \theta  } 
     \\
     &= \cos \theta + i \sin \theta,
     \end{align*}
     as required.
 \item[\ding{50} \ding{175}] Let $z \in  \CC  $ and write
   \[
   z = x + iy \quad \quad (x, y \in  \RR) .
   \]
   we have
   \begin{align*}
     e^{z} &= e^{x +  i y} 
        \\ & \overset{(1) }{=} e^{x} \cdot  e^{iy} \\
           & \overset{(3) }{=} e^{x}(\cos y + i \sin y    )  \\
           &= e^{x}\cos y + i e^{x} \sin  y.    
   \end{align*}
   Thus
   \begin{align*}
     e^{z } = 1 &\iff 
   \begin{cases}
   e^{x} \cos y = 1 \\
   e^{x} \sin  y = 0  
   \end{cases}
    \iff 
   \begin{cases}
   \cos y = e^{-x} > 0 \\
   \sin y = 0   
   \end{cases} \\
   & \iff 
   \begin{cases}
   \exists k \in   \mathbb{Z}: \quad y = 2 \pi  k \\
   e^{-x} = \cos 2 \pi  k  = 1   
   \end{cases} 
    \iff \begin{cases}
   \exists  k \in   \mathbb{Z}: \quad y = 2\pi  k \\
   x = 0 
   \end{cases} \\
   &\iff z = 2\pi  k i \quad \quad ( k \in  \mathbb{Z})  \\
   & \iff z \in   2 \pi  \mathbb{Z},
   \end{align*}
   as required.
\end{proof}
\subsection{Trigonometric and hyperbolic functions}
\begin{definition}[Complex Trigonometric functions]
  We define the trigonometric functions cosine and sine by:
  \begin{align*}
    \cos z &:= \sum_{n=0}^{+\infty} (-1) ^n \frac{z^{2n}}{(2n) !},   \\
    \sin z &:= 
    \sum_{n=0}^{+\infty} (-1) ^n \frac{z^{2n+1}}{(2n+1) !} \quad \quad (\forall  z \in   \CC ).
  \end{align*}
  Clearly, these functions extend the real functions cos and sin. The power series defining cos and sin have infinite
  radius of convergence, thus (By a previous theorem) cos and sin are analytic on $\CC  $; that is, cos and sin are 
  \underline{
  entire functions.
  }
\end{definition}
\begin{remark}
  We easily verify the extended Euler's formula: 
  \[
  e^{iz} = \cos z + i \sin z \quad \quad ( \forall  z \in   \CC ) .    
  \]
  From this formula, we derive: 
  \begin{align*}
    \cos z &= \frac{e^{iz} + e^{-iz}}{2}, \\
    \sin z &= \frac{e^{iz} - e^{-iz}}{2i} \quad \quad (\forall  z \in   \CC ).
  \end{align*}
\end{remark}
  \marginpar{These functions are not bounded in $\CC  $, when you replace $x \leftarrow ix $, you get 
  $\cos ix   = \cosh x   $.}
\exercise
Using property \ding{175} of \it Proposition 1.3.2 \normalfont and Euler's formula, show the following 
properties:
\begin{itemize}
  \item[\ding{172}] The functions $\cos    $ and $\sin    $ are both $2\pi  $-periodic.
 \item[\ding{173}] The set of zeros of $z \mapsto \cos z   $ is $(\frac{\pi }{2} + \pi  \mathbb{Z})  $, while
   the set of zeros of $ z \mapsto \sin z   $ is $\pi  \mathbb{Z}$. 
 \item[\ding{174}] For all $z \in  \CC  $, we have
   \[
   \cos ^2 z + \sin ^2  z = 1  .
   \]
\end{itemize}
\underline{\ding{43} \textsc{For Example, for \ding{174}: }}By the Euler formula, we have for all $z \in  \CC  $: 
\begin{align*}
  \cos ^2  z + \sin ^2  z &= 
  \left( \frac{e^{iz} + e^{-iz}}{2} \right) ^2  + 
  \left( \frac{e^{iz} - e^{-iz}}{2i} \right) ^2  \\
                          &= \frac{4}{4} = 1 
\end{align*}
\begin{definition}[Complex hyperbolic functions]
  We define the hyperbolic functions cosh and sinh by: 
  \begin{align*}
    \cosh z &:= 
    \sum_{n=0}^{+\infty} \frac{z^{2n}}{(2n) !} 
    = \frac{e^{z} + e^{-z}}{2} = \cos (iz) , \\
    \sinh z &:=
    \sum_{n=0}^{+\infty} \frac{z^{2n+1}}{(2n+1) !} =
    \frac{e^{z} - e^{-z}}{2} = - i \sin (iz)  \quad \quad 
    (\forall z \in   \CC ).
  \end{align*}
  Clearly, these definitions extend the real functions cosh and sinh. Like the trigonometric functions
  cos and sin, the hyperbolic functions cosh and sinh are also \underline{entire functions.}
\end{definition}
\exercise
Using the expressions of cosh and sinh in terms of cos and sin, verify the following properties:
\begin{itemize}
  \item[\ding{172}] The functions cosh and sinh are both $2\pi-$periodic.
   \item[\ding{173}] The set of zeros of cosh is 
     $(\frac{\pi }{2}i + \pi i \mathbb{Z})$, while the set of zeros of sinh is $\pi  i \mathbb{Z} $ .
    \item[\ding{174}] For all $z \in  \CC  $, we have
      \[
        \cosh ^2 z - \sinh ^2  z = 1.
      \]
\end{itemize}
\begin{definition}[Further trigonometric and hyperbolic functions]
  We define the following functions:
  \begin{align*}
    \tan z &:= \frac{\sin z  }{\cos z  } \quad \quad 
    \left( 
    \forall z \in  \CC \backslash 
    \left( 
   \frac{\pi }{2} + \pi \mathbb{Z} 
    \right) 
    \right),
    \\
    \cot z  &:= \frac{\cos  z}{\sin z  } \quad \quad 
    \left( 
    \forall z \in  \CC \backslash \pi  \mathbb{Z} 
    \right),
    \\
    \tanh z &:= \frac{\sinh z  }{\cosh z  }   \quad \quad 
    \left( 
    \forall z \in   \CC \backslash 
      \left( 
    \frac{\pi }{2} i + 
    \pi i \mathbb{Z}
      \right) 
    \right), 
    \\
    \coth z &:= \frac{\cosh z  }{\sinh z  }\quad \quad (\forall z \in  \CC  \backslash  \pi  i \mathbb{Z}) .
  \end{align*}
  This clearly extends the well-known real functions tan, cot, tanh, and coth. Note that each of these four functions
  is analytic in its domain of definition (according to the previous results on analytic functions).
\end{definition}
\section{Holomorph functions}
\begin{definition}[]
Let $\Omega  $ be a nonempty open subset of $\CC  $ and $z_0 $ be a point in $\Omega  $. Let also
$ f : \Omega  \longrightarrow \CC  $ be a map. 
\begin{itemize}
  \item 
We say that $f $ is \underline{holomorphic} at $z_0 $ 
if the limit
\[
  \lim_{z \to z_0} \frac{f(z)  - f(z_0) }{z-z_0}
\]
exists and belong to $\CC$. In this case, the limit is called the \underline{derivative} of $f $ at the point
$z_0 $ and denoted by $f'(z_0)  $. 
\end{itemize}
\begin{itemize}
  \item We say that $f $ is \underline{holomorphic on $\Omega  $ } if it is holomorphic at every point 
    in $\Omega  $. \\ In this case, the function
    \[
    \begin{array}{cccc}
          f' : &  \Omega   & \longrightarrow & \CC  \\
    
               &  z  & \longmapsto     & f'(z)  \\ 
    \end{array}
    \]
    is called the derivative of $f$. 
\end{itemize}
\end{definition}
\begin{proposition}[Holomorphy of power series]
  Let $z_0 \in  \CC  $, $(a_n ) _{n \in   \NN_0} \subset \CC $, and $S $ be the power series
  \[
  S(z)  = \sum_{n=0}^{+\infty} a_n (z-z_0) ^n.
  \]
  Suppose that $S $ has a positive radius of convergence $R $. Then $S $ is holomorphic on 
  $D(z_0, R)  $ and we have for all $z \in  D(z_0, R):$
  \begin{align*}
    S'(z)  &= \sum_{n=0}^{+\infty} n a_n (z-z_0) ^{n-1} \\
           &= \sum_{n=0}^{+\infty} (n+1) a_{n+1} (z-z_0) ^n .
  \end{align*}
\end{proposition}
\begin{proof}
For simplicity, suppose without loss of generality that $z_0 = 0 $. First, remark that 
by using the Hadamard formula, the power series
\[
\sum_{n=1}^{+\infty}  n a_n z^{n-1} = \sum_{n=0}^{+\infty} (n+1) a_{n+1} 
z^n 
\]
has the same radius of convergence $R $ as $S $. It follows that $\sum_{n=1}^{+\infty} n a_n z^{n-1}$ is absolutely
convergent on $D(0, R)$; That is, for all $0 <  r < R$, the series $\sum_{n=1}^{+\infty} n \left| a_n  \right|  z^{n-1}$  converges. 
Now, let $z_1 \in  D(0, R)$ be arbitrary and show that $S $ is holomorphic at $z_1.$ Choose 
$r \in  \RR  $ such that $\left| z_1  \right|  <  r < R$. For all $z \in   D(0, r)  \backslash  \left\{ z_1 \right\}$, we have
\begin{align*}
  \frac{S(z) - S(z_1)  }{z - z_1} &= 
  \frac{\sum_{n=0}^{+\infty} a _n z^n  - \sum_{n=0}^{+\infty} a_n z_1^n }{ z - z_1} \\
                                  &= 
                                  \sum_{n=0}^{+\infty} a_n  \frac{z^n - z_1^n }{z - z_1} \\
                                  &= 
                                  \sum_{n=1}^{+\infty} a_n 
                                  \sum_{k=0}^{n-1} z^{k} z_1^{n-1-k}.
\end{align*}
\end{proof}
% end of file
